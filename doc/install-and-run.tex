\section{Installing GCP}

Binary releases are available for release versions of GCP starting with
version 2.5, as a ZIP file which can be downloaded from
GitHub at \url{https://github.com/GateNLP/gcp/releases}\footnote{Versions
prior to 3.0 are available from SourceForge at
\url{http://sf.net/projects/gate/files/gcp/}.}.  For development versions the
software must be built from source.  The source code is available on GitHub
at \url{https://github.com/GateNLP/gcp}.

To build GCP you will need a Java 8 JDK.  Sun/Oracle and OpenJDK have been
tested and are known to work.  GCJ is known {\em not} to work.  You will also
need Apache Maven 3.3 or later.  Running ``mvn install'' will build the various
components of GCP and create a ZIP file containing the binary distribution
under \verb!distribution/target!.  Unpack that file somewhere to create your
GCP installation.

\section{Running GCP}

Once GCP is installed you can run it in one of two ways:
\bit
\item using the \verb!gcp-cli.jar! executable
JAR file in the installation directory.
\item using the \verb!gcp-direct.sh! bash script.
\eit

\subsection{Using {\tt gcp-cli.jar}}

The usual way to run GCP is to write one or more {\em batch definition} XML
files (see chapter~\ref{chap:batch-def} for details) defining the application
you want to run, the documents to process, and the output formats to produce.
You then pass these batch definitions to \verb!gcp-cli.jar! for processing.
The CLI tool takes a number of optional arguments:

\bde
\item[-m] Specifies the maximum Java heap size, in the format expected by the
  usual \verb!-Xmx! Java option, e.g. \verb!-m 10G! for a 10GB heap limit.  The
  default setting is \verb!12G!.  The \verb!gcp-cli! will spawn a separate java
  process to run each batch, passing this memory limit to that process.  This
  is different from specifying a \verb!-Xmx! option to \verb!gcp-cli!, which
  would define the heap size limit for the CLI process itself, not the batch
  runner processes it spawns.
\item[-t] Specifies the number of threads that GCP should use to execute the
  GATE application.  Typically this should be set to between 1 and 1.5 times
  the number of processing cores available on the machine.  The default value
  is 6, which is generally suitable for a 4-core machine.
\item[-D] Java system property settings, for example
  \verb!-Djava.io.tmpdir=/home/bigtmp!.  \verb!-D! options specified before the
  \verb!-jar! apply to the virtual machine running the CLI, those specified
  after \verb!-jar gcp-cli.jar! will be passed to the batch runner processes.
\ede

GATE plugins can also be pre-loaded using the \verb!-C! and \verb!-p! options,
see the ``gcp-direct'' section below for details on these.

The tool will determine the location of where GCP is installed in the 
following order, taking the first it can find:
\ben
\item The value of the environment variable \verb!GCP_HOME! if it is set.
\item The value of the propery  \verb!gcp.home! if it is set.
\item The location of the JAR file used for running the program.
\een
Note that if the environment variable \verb!GCP_HOME! is set to a different
directory than the one used to run gcp-cli, the version of 
the batch runner in the directory pointed to by \verb!GCP_HOME! will get invoked
which is probably not what is intended.

The settings for the \verb!-m! and \verb!-t! options are typically a trade-off
-- if your application is particularly memory-hungry or you are processing
particularly large or complex documents you may need to lower the number of
processing threads in order to give more memory (on average) to each one.

GCP can run in two modes.  In the basic ``single-batch'' mode the final
command-line argument is simply the path to a single {\em batch definition} XML
file (see chapter~\ref{chap:batch-def} for details), and GCP will process that
batch and then exit.

The other (and more commonly used) mode is ``multi-batch'' mode, signified by
the \verb!-d! command line option.  In this mode the final command-line
argument is the path to a directory referred to as the {\em working directory}.
\begin{verbatim}
java -jar gcp-cli.jar -t 4 -m 8G -d /data/gcp
\end{verbatim}

The working directory is expected to contain a subdirectory named ``in'', and
any file in this directory with the extension \verb!.xml! (in lower case) is
assumed to be a batch definition file.  For each batch {\tt {\it batch}.xml} in
the ``in'' directory, the script will:

\bit
\item run the batch, redirecting the standard output and error streams to a
  file {\tt {\it working-dir}/logs/{\it batch}.xml.log}
\item if the batch completes successfully, move the definition file to
  {\tt {\it working-dir}/out/{\it batch}.xml}
\item or, if the batch fails (i.e. the Java process exits with a non-zero exit
  code, which occurs if, for example, one of the processing threads encounters
  an OutOfMemoryError), move the definition file to
  {\tt {\it working-dir}/err/{\it batch}.xml}
\eit

Additional batches can be added to the ``in'' directory at any time -- whenever
a batch completes the script will re-scan the ``in'' directory to locate the
next available batch.  In particular, failed batches can be moved back from
``err'' to ``in'' and they will be re-processed, and if the report file for the
failed batch is intact GCP will continue on from where it left off on the
previous run.

Creating a file named \verb!shutdown.gcp! in the ``in'' directory will cause
the script to exit at the end of the batch it is currently processing (or
immediately if it is currently idle).


\subsection{Using {\tt gcp-direct.sh}}
\label{sec:running:gcp-direct}

The \verb!gcp-direct.sh! script can be used for simple cases where you want to
process all the files under one particular directory and output the resulting
annotations in GATE XML or FastInfoset format.  For this specific case it is
not necessary to write an XML batch descriptor, you can specify the required
parameters using command line options to \verb!gcp-direct.sh!:

\bde
\item[-t] the number of parallel threads to use.
\item[-x] the path to the saved GATE application that you want to run.
\item[-f] the output format to use for saving results, must be either ``xml''
  (GATE XML format) or ``finf'' (FastInfoset format).  To use FastInfoset the
  GATE \verb!Format: FastInfoset! plugin must be loaded by the saved
  application or by using \verb!-p uk.ac.gate.plugins:format-fastinfoset:8.5!
  (see below).
\item[-i] the directory in which to look for the input files or a file that contains
  relative path names to the input files. If this points to a directory, all files in
  this directory and any subdirectories will be processed (except for standard
  backup and temporary file name patterns and source control metadata -- see
  \url{http://ant.apache.org/manual/dirtasks.html#defaultexcludes} for
  details). If this points to a file, the content of the file is expected to be 
  one relative file path per line, using UTF-8 encoding. The file paths are 
  interpreted to be relative to the directory that contains the list file.
  If processed documents are written, then this will also be their relative 
  path to the output directory. 
\item[-o] (optional) the directory in which to place the output files.  Each input file
  will generate an output file with the same name in the output directory.
  If this option is missing, and the option \texttt{-b} is missing as well,
  the documents are not saved!
\item[-b] (optional) if this option is specified it can be used to specify
  a batch file. In that case, the options \texttt{-x, -i, -o, -r, -I} are 
  not required and ignored if specified and the corresponding information is
  taken from the batch configuration file instead.
\item[-r] (optional) path to the report file for this batch -- if omitted
  GCP will use \verb!report.xml! in the current directory.
\item[-ci] the input files are all gzip-compressed
\item[-co] the output files should all be gzip-compressed. This only makes
sense if \verb!-f xml! is also specified since the default output format
\verb!finf! already is a compressed format. If this option is specified, the
output file name gets the extension \verb!.gz! appended, in addition to 
any other extension it already may have.
\item[-p] (optional, may be specified multiple times) a GATE plugin to pre-load
  in addition to those specified by the saved application.  The value of this
  option can be one of three formats (tried in this order):
  \ben
  \item a set of Maven co-ordinates \verb!group:artifact:version!, to load
    a plugin from a Maven repository.
  \item an absolute URL e.g. starting \verb!file:/...! or \verb!http://!, to
    load a directory-based plugin from that URL.
  \item a local file path, either absolute or relative to the working directory
    of the GCP process, to load a directory-based plugin from disk.
  \een
\item[-C] (optional, may be specified multiple times) when loading Maven-style
  plugins using \verb!-p! GATE will typically go out to the internet to fetch the
  plugin and its dependencies from a remote Maven repository.  If you have a local
  Maven cache of plugins on your disk you can specify its location with this option
  and the local cache will be searched first before attempting to download plugins
  from the network.  If you wish to completely disable loading of dependencies from
  the network (e.g. if you are running on a machine that does not have internet
  access) you will need to provide a custom Maven settings file as detailed in
  section~\ref{sec:maven-settings}.
\ede

Additionally, you can specify \verb!-D! and \verb!-X! options which will be
passed through to the Java VM, for example you can set the maximum amount of
heap memory that the JVM can use with an option like \verb!-Xmx2G!

The \verb!gcp-direct.sh! script is deliberately opinionated, in order to reduce
the number of different options that need to be set, and it has a number of
hard-coded assumptions.  It assumes that your input documents use the UTF-8
character encoding, that the correct document format parser to use can be
determined from the file extension, and that you always want to save \emph{all}
the annotations that your application generates.  If you need to process
documents in a different encoding, you have more complex output requirements
(XCES, JSON, M\'{i}mir, \ldots) or want to output only a subset of the GATE
annotations from each document, then you should write a batch definition in XML
and use \verb!gcp-cli.jar! as discussed above.

% vim:ft=tex
