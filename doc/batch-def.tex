\section{The Structure of a Batch Descriptor}

GCP batches are defined by an XML file whose format is as follows.  The root
element defines the batch identifier:
\begin{lstlisting}[language=XML]
<batch id="batch-id" xmlns="http://gate.ac.uk/ns/cloud/batch/1.0">
\end{lstlisting}

The children of this \verb!<batch>! element are:
\bde
\item[application] (required) specifies the location of the saved GATE
  application state. \verb!<application file="../annie.xgapp"/>!

\item[report] (required) specifies the location of the XML report file.  If the
  report file already exists GCP will read it and process only those documents
  that have not already been processed successfully.
  \verb!<report file="../report.xml" />!

\item[input] (required) specifies the input handler which will be the source of
  documents to process.  Most handlers load documents one by one based on their
  IDs, but certain handlers operate in a \emph{streaming} mode, processing a
  block of documents in one pass.

\item[output] (zero or more) specified what to do with the documents once they
  have been processed.

\item[documents] (required, except when using a streaming input handler)
  specifies the document IDs to be processed, as any combination of the child
  elements:
  \bde
  \item[id] a single document ID. \verb!<id>bbc/article001.html</id>!
  \item[documentEnumerator] an enumerator that generates a list of IDs.  The
    enumerator implementation chosen will typically depend on the specific type
    of input handler that the batch uses.
  \ede
\ede

The following example shows a simple XML batch definition file which runs ANNIE
and saves the results as GATE XML format.  The input, output and documents
elements are discussed in more detail in the following sections.

\begin{lstlisting}[language=XML, escapeinside={~}{~}]
<?xml version="1.0" encoding="UTF-8"?>
<batch id="sample" xmlns="http://gate.ac.uk/ns/cloud/batch/1.0">
  ~\label{lst:sample:app}~<application file="../annie.xgapp"/>

  ~\label{lst:sample:report}~<report file="../reports/sample-report.xml" />

  ~\label{lst:sample:input-start}~<input dir="../input-files"
         mimeType="text/html"
         compression="none"
         encoding="UTF-8"
  ~\label{lst:sample:input-end}~       class="gate.cloud.io.file.FileInputHandler" />

  ~\label{lst:sample:output-start}~<output dir="../output-files-gate"
          compression="gzip"
          encoding="UTF-8"
          fileExtension=".GATE.xml.gz"
  ~\label{lst:sample:output-end}~        class="gate.cloud.io.file.GATEStandOffFileOutputHandler" />

  ~\label{lst:sample:documents}~<documents>
    <id>ft/03082001.html</id>
    <id>gu/04082001.html</id>
    <id>in/09082001.html</id>
  </documents>
</batch>
\end{lstlisting}

It is important to note that all relative file paths specified in a batch
descriptor are resolved against the location of the descriptor file itself,
thus if this descriptor file were located at
\verb!/data/gcp/batches/sample.xml! then it would load the application from
\verb!/data/gcp/annie.xgapp!.


\section{Specifying the Input Handler}

Each batch definition must include a single \verb!<input>! element defining the
source of documents to be processed.  Given a document ID, the job of the input
handler is to locate the identified document and load it as a
\verb!gate.Document! to be processed by the application.  Note that the input
handler describes how to find the document for each ID but does {\em not}
define which IDs are to be processed, that is the job of the \verb!<documents>!
element below.

The \verb!<input>!  element must have a \verb!class! attribute specifying the
name of the Java class implementing the handler.  GCP will create an instance
of this class and pass the remaining \verb!<input>! attributes to the handler
to allow it to configure itself.  Thus, which attributes are supported and/or
required depends on the specific handler class.

GCP provides four standard input handler types:

\bit
\item \verb!gate.cloud.io.file.FileInputHandler! to read documents from
  individual files on the filesystem
\item \verb!gate.cloud.io.zip.ZipInputHandler! to read documents directly from
  a ZIP archive
\item \verb!gate.cloud.io.arc.ARCInputHandler! and
  \verb!gate.cloud.io.arc.WARCInputHandler! to read documents from an ARC or
  WARC archive as produced by the Heritrix web crawler
  (\url{http://crawler.archive.org}).
\eit

and one \emph{streaming} handler:

\bit
\item \verb!gate.cloud.io.json.JSONStreamingInputHandler! to read a stream of
  documents from a single large JSON file (for example a collection of Tweets
  from Twitter's streaming API).
\eit

\subsection{The {\tt FileInputHandler}}

\verb!FileInputHandler! reads documents from individual files on the
filesystem.  It can read any document format supported by GATE Embedded, and in
addition it can read files that are GZIP compressed, unpacking them on the fly
as they are loaded.  It supports the following attributes on the \verb!<input>!
element in the batch descriptor:

\bde
\item[encoding] (optional) The character encoding that should be used to read
  the documents (i.e. the value for the encoding parameter when creating a
  \verb!DocumentImpl! using the GATE Factory).  If omitted, the default GATE
  Embedded behaviour applies, i.e. the platform default encoding is used.
\item[mimeType] (optional) The MIME type that should be assumed when creating
  the document (i.e. the value of the \verb!DocumentImpl! mimeType parameter).
  If omitted GATE Embedded will attempt to guess the appropriate MIME type for
  each document in the usual way, based on the file name extension and magic
  number tests.
\item[compression] (optional) The compression that has been applied to the
  files, either ``none'' (the default) or ``gzip''.
\ede

The actual mapping from document IDs to file locations is controlled by a
{\em naming strategy}, another Java object which is configured from the
\verb!<input>! attributes.  The default naming strategy
({\tt gate.cloud.io.file.SimpleNamingStrategy}) treats the document ID as a
relative path\footnote{Technically a relative {\em URI}, so forward slashes
must be used in document IDs even when running on Windows where file paths
normally use backslashes.}, and takes the following attributes:

\bde
\item[dir] (required) The base directory under which documents are found.
\item[fileExtension] (optional) A file extension to append to the document ID.
\ede

Given a document ID such as ``ft/03082001'', a base directory of ``/data'' and
a file extension of ``.html'' the \verb!SimpleNamingStrategy! would load the
file ``/data/ft/03082001.html''

To use a different naming strategy implementation, specify the Java class name
of the custom strategy class as the \verb!namingStrategy! attribute of the
\verb!<input>! element, along with any other attributes the strategy requires
to configure it.

\subsection{The {\tt ZipInputHandler}}

The ZIP input handler reads documents directly out of a ZIP archive, and is
configured in a similar way to the file-based handler.  It supports the
following attributes:

\bde
\item[encoding] (optional) exactly as for \verb!FileInputHandler!
\item[mimeType] (optional) exactly as for \verb!FileInputHandler!
\item[srcFile] (required) The location of the ZIP file from which documents
  will be read. This parameter was previously named ``zipFile'', the old name
  is supported for backwards compatibility but not recommended for new batches.
\item[fileNameEncoding] (optional) The default character encoding to assume for
  file names inside the ZIP file.  This attribute is only relevant if the ZIP
  file contains files whose names contain non-ASCII characters {\em without}
  the ``language encoding flag'' or ``Unicode extra fields'', and can be
  omitted if this does not apply.  There is a detailed discussion on file name
  encodings in ZIP files in the Ant manual
  (\url{http://ant.apache.org/manual/Tasks/zip.html#encoding}), but the rule of
  thumb is that if the ZIP file was created using Windows ``compressed
  folders'' then \verb!fileNameEncoding!  should be set to match the encoding
  of the machine that created the ZIP file, otherwise the correct value is
  probably ``Cp437'' or ``UTF-8''.
\ede

The ZIP input handler does not use pluggable naming strategies, and simply
assumes that the document ID is the path of an entry in the ZIP file.

\subsection{The {\tt ARCInputHandler} and {\tt WARCInputHandler}}
\label{sec:batch-def:arc}

These two input handlers read documents out of ARC- and WARC format web archive
files as produced by the Heritrix web crawler and other similar tools.  They
support the following attributes:

\bde
\item[srcFile] (optional) The location of the archive file\footnote{For ARC,
  this parameter was previously called ``arcFile'', the old name is supported
  for backwards compatibility but not recommended for new batches.}.  These
  input handlers can operate in one of two modes -- if \verb!srcFile! \emph{is}
  specified then the handler will load records from this specific archive file
  on disk, but if \verb!srcFile! is \emph{not} specified then each document ID
  must provide a fully qualified http or https URL to an archive.  In the
  second mode the selected records will be downloaded individually using ``byte
  range'' HTTP requests.
\item[defaultEncoding] (optional) The {\em default} character encoding to
  assume for entries that do not specify their encoding in the entry
  headers.  If an entry specifies its own encoding explicitly this will be
  used.  If this attribute is omitted, ``Windows-1252'' is assumed as the
  default.
\item[mimeType] (optional) The MIME type that should be assumed when creating
  the document (i.e. the value of the \verb!DocumentImpl! mimeType parameter).
  If omitted, the usual GATE Embedded heuristics will apply.  The input
  handlers make the HTTP headers from the archive entry available to GATE as if
  the document had been downloaded directly from the web, so the
  \verb!Content-Type! header from the archive entry is available to these
  heuristics.
\ede

The web archive input handlers expect document IDs of the following form:
\begin{lstlisting}[language=XML]
<id recordPosition="NNN" [url="optional url of archive"]
    recordOffset="NNN" recordLength="NNN">{original entry url}</id>
\end{lstlisting}

The content of the \verb!id! element should be the original URL from which the
entry was crawled, and the attributes are:
\bde
\item[recordPosition] a numeric value that
 is used as a sequence number. If the IDs are generated by the corresponding
 enumerator (see below), then the this attribute will contain the actual
 record position inside the archive file.
\item[recordOffset and recordLength] the byte offset of the required record in
  the archive, and the record's length in bytes.
\item[url] (optional) a full HTTP or HTTPS URL to the source archive file.  If
  this is provided, GCP will download just the specific target record using a
  ``Range'' header on the HTTP request, rather than loading the record from the
  input handler's usual \verb!srcFile!.
\ede

The standard enumerator implementations (see below) create IDs in the correct
form.

The ARC input handler adds all the HTTP headers and archive record headers for
the entry as features on the GATE \verb!Document! it creates.  HTTP header
names are prefixed with ``http\_header\_'' and ARC/WARC record headers with
``arc\_header\_''.

\subsection{The streaming JSON input handler}
\label{sec:batch-def:json-input}

An increasing number of services, most notably Twitter and social media
aggregators such as DataSift, provide their data in JSON format.  Twitter
offers streaming APIs that deliver Tweets as a continuous stream of JSON
objects concatenated together, DataSift typically delivers a large JSON array
of documents.  The streaming JSON input handler can process either format,
treating each JSON object in the ``stream'' as a separate GATE document.

The \verb!gate.cloud.io.json.JSONStreamingInputHandler! accepts the following
attributes:

\bde
\item[srcFile] the file containing the JSON objects (either as a top-level
  array or simply concatenated together, optionally separated by whitespace).
\item[idPointer] the ``path'' within each JSON object of the property that
  represents the document identifier.  This is an expression in the \emph{JSON
  Pointer}\footnote{\url{http://tools.ietf.org/html/draft-ietf-appsawg-json-pointer-03}}
  language.  It must start with a forward slash and then a sequence of property
  names separated by further slashes.  A suitable value for the Twitter JSON
  format would be \verb!/id_str! (the property named ``\verb!id_str!'' of the
  object), and for DataSift \verb!/interaction/id! (the top-level object has an
  ``interaction'' property whose value is an object, we want the ``id''
  property of \emph{that} object).  Any object that does not have a property at
  the specified path will be ignored.
\item[compression] (optional) the compression format used by the
  \verb!srcFile!, if any.  If the value is ``none'' (the default) then the file
  is assumed not to be compressed, if the value is one of the compression formats
  supported by Apache Commons Compress (``gz''\footnote{For backwards
  compatibility, ``gzip'' is treated as an alias for ``gz''}, ``bzip2'',
  ``xz'', ``lzma'', ``snappy-raw'', ``snappy-framed'', ``pack200'', ``z'') then 
  it will be unpacked using that library.  If the value is ``any'' then the
  handler uses the auto-detection capabilities of Commons Compress to attempt
  to detect the appropriate compression format.  Any other value is taken to be
  the command line for a native decompression program that expects compressed
  data on stdin and will produce decompressed data on stdout, for example
  \verb!"lzop -dc"!.
\item[mimeType] (optional but highly recommended) the value to pass as the
  ``mimeType'' parameter when creating a GATE Document from the JSON string.
  This will be used by GATE to select an appropriate document format parser, so
  for Twitter JSON you should use \verb!"text/x-json-twitter"! and for DataSift
  \verb!"text/x-json-datasift"!.  Note that the GATE plugin defining the
  relevant format parser \emph{must} be loaded as part of your GATE
  application.
\ede

This is a streaming handler -- it will process all documents in the JSON bundle
and does \emph{not} require a \verb!documents! section in the batch
specification.  As with other input handlers, when restarting a failed batch
documents that were successfully processed in the previous run will be skipped.

\section{Specifying the Output Handlers}

Output handlers are responsible for taking the GATE Documents that have been
processed by the application and doing something with the results.  GCP
supplies a number of standard output handlers to save the document text and
annotations to files in various formats, and also a handler to send the
annotated documents to a remote M\'{i}mir server for indexing.

Most batches would specify at least one output handler but GCP does support
batches with no outputs (if, for example, the application itself contains a PR
responsible for outputting results).

Output handlers are specified using \verb!<output>! elements in the batch
definition, and like input handlers these require a \verb!class! attribute
specifying the implementing Java class name.  Other attributes are passed to
the instantiated handler object to allow it to configure itself.

By default, an output handler will save all annotations from all annotation
sets in each document.  A given output handler may be configured to save only a
subset of the annotations by providing \verb!<annotationSet>! sub-elements
inside the \verb!<output>! element, for example

\begin{lstlisting}[language=XML, escapeinside={~}{~}]
<annotationSet name="ANNIE">
  <annotationType name="Person" />
  <annotationType name="Location" />
</annotationSet>

~\label{lst:annsets:default}~<annotationSet />
\end{lstlisting}

The \verb!<annotationSet>! element may have a \verb!name! attribute giving the
annotation set name (if omitted the default annotation set is used), and zero
or more \verb!<annotationType>! sub-elements giving the annotation types to
extract from that set (if no \verb!<annotationType>! elements are provided, all
annotation from the set are saved, so line~\ref{lst:annsets:default} specifies
that the handler should save all annotations from the default set).

Note that these filters are provided as a convenience, and some output handler
implementations may ignore them.  For example the M\'{i}mir output handler
always sends the complete Document to the M\'{i}mir server, regardless of the
filters specified.

\subsection{File-based Output Handlers}\label{sec:batch-def:file-output-handlers}

GCP provides a set of six standard file-based output handlers to save data to
files on the filesystem in various formats.

\bit
\item \verb!gate.cloud.io.file.GATEStandOffFileOutputHandler! to save documents
  in the GATE XML format (``save as XML'' in GATE Developer).
\item \verb!gate.cloud.io.file.GATEInlineOutputHandler! to save documents with
  inline XML tags for their annotations (``save preserving format'' in GATE
  Developer).
\item \verb!gate.cloud.io.file.PlainTextOutputHandler! to save just the text
  content of the document.  This is rarely useful on its own but is frequently
  used in conjunction with
\item \verb!gate.cloud.io.xces.XCESOutputHandler! to save annotations in the
  XCES standoff format.  Annotation offsets in XCES refer to the plain text as
  saved by a \verb!PlainTextOutputHandler!.
\item \verb!gate.cloud.io.file.JSONOutputHandler! to save documents in a JSON
  format modelled on that used by Twitter to represent ``entities'' in Tweets.
\item \verb!gate.cloud.io.json.JSONStreamingOutputHandler! saves documents in
  the same JSON format as the previous handler, but concatenated together in
  one or more output batches rather than saving each document in its own
  individual output file.
\item \verb!gate.cloud.io.file.SerializedObjectOutputHandler! to save documents
  using Java's built in \emph{object serialization} protocol (with optional
  compression).  This handler ignores annotation filters, and always writes
  the complete document.  This is the same mechanism used by GATE's
  \verb!SerialDataStore!.
\eit

The handlers share the following \verb!<output>! attributes:

\bde
\item[encoding] (optional, not applicable to
  \verb!SerializedObjectOutputHandler!) The character encoding used when
  writing files.  If omitted, ``UTF-8'' is the default.
\item[compression] (optional) The compression algorithm to apply to the saved
  files.  Can be either ``none'' (no compression, the default) or ``gzip''
  (GZIP compression).
\ede

As with the file-based input handler, these output handlers use a {\em naming
strategy} to map from document IDs to output file names.  The default strategy
is the same \verb!SimpleNamingStrategy! configured with a base \verb!dir! and a
\verb!fileExtension!, treating the document ID as a path relative to the given
directory and appending the given extension. If the \verb!replaceExtension! parameter
is set to \verb!"true"! then the \verb!fileExtension!, if specified, replaces
any existing file extension of the intput path.

This is appropriate when using a
file or ZIP input handler but for batches that use an \verb!ARCInputHandler! a
different strategy is required.

As document IDs for an \verb!ARCInputHandler! are based on URLs the simple
strategy would try to put the output files into directories named after
absolute URLs, which can include characters that are not permitted in file
names on all platforms.  An alternative strategy is provided that makes use of
the \verb!recordPosition! attribute on the IDs to put output files into a
hierarchy of numbered directories.  To use this strategy, specify an attribute
\verb!namingStrategy="gate.cloud.io.arc.ARCDocumentNamingStrategy"!, and the
usual \verb!dir! and \verb!fileExtension! attributes of the default strategy.
The ARC strategy also accepts an optional additional attribute \verb!pattern!
defining the pattern to use to map the ID number to a directory.

The default pattern is ``3/3'', which will left-pad the \verb!recordPosition!
to a minimum of 6 digits and then create one level of directories from the
first three digits and use the last three as part of the file name\footnote{In
fact the pattern is processed from right to left, so any surplus digits end up
in the first place, i.e. the ID 1234567 becomes 1234/567 rather than
123/4567.}.  The ID text (i.e. the original URL) is cleaned up to remove the
protocol, query string and fragment (if any) and replace slash and colon
characters with underscores (so the resulting file name will not include any
more levels of subdirectories) and appended to the numeric part following an
underscore.  For full details of this process, see the JavaDoc
documentation.  As an example, the ID with \verb!recordPosition="1"! and URL
\verb!http://example.com/file.html! with the default pattern of ``3/3'' would
map to the target path ``000/001\_example.com\_file.html'', and this
would then be combined with the \verb!dir! and \verb!fileExtension! to produce
the final file name.

The \verb!PlainTextOutputHandler! simply saves the plain text of the GATE
document with no annotations (so \verb!<annotationSet>! filters are ignored).
The \verb!GATEStandOffFileOutputHandler! writes the document text and selected
annotations in the standard ``save as XML'' GATE XML format.  The
\verb!XCESOutputHandler! saves the selected annotations as XCES XML format.

The \verb!GATEInlineOutputHandler! saves the document text plus selected
annotations as inline XML tags as produced by ``save preserving format'' in
GATE Developer.  This handler supports one additional \verb!<output>! attribute
named \verb!includeFeatures! -- if this is set to ``true'', ``yes'' or ``on''
then the annotation features will be included as attributes on the XML tags,
otherwise (including if the attribute is omitted) it will save just the tags
with no attributes.

The \verb!JSONOutputHandler! saves the document in a JSON format modelled on
that used by Twitter to represent entities in Tweets.  This is a JSON object
with two properties, ``text'' holding the plain text of the document and
``entities'' holding the annotations.  The ``entities'' value is itself an
object mapping a ``label'' to an array of annotations.
%
\begin{verbatim}
{
  "text":"The text of the document",
  "entities":{
    "Person":[
      {
        "indices":[start,end],
        "feature1":"value1",
        "feature2":"value2"
      },
      {
        "indices":[start,end],
        "feature1":"value1",
        "feature2":"value2"
      }
    ]
  }
}
\end{verbatim}

For each annotation the ``indices'' property gives the start and end offsets of
the annotation as character offsets into the ``text'', and the other properties
of the object represent the features of the annotation.

This handler supports a number of additional \verb!<output>! attributes to
control the format.

\bde
\item[groupEntitiesBy] controls how the annotations are grouped under the
  ``entities'' object.  Permitted values are ``type'' (the default) or ``set''.
  Grouping by ``type'' produces output like the example above, with one entry
  under ``entities'' for each annotation type containing all annotations of
  that type from across all annotation sets that were selected by the
  \verb!<annotationSet>! filters.  Conversely, grouping by ``set'' creates one
  entry under ``entities'' for each annotation set name (with the name
  ``default'' used for the default annotation set -- technically JSON
  permits the empty string as a property name but this is likely to cause
  problems for some consumer libraries), containing all the annotations in
  that set that were selected by the filters, regardless of type.  Grouping by
  ``set'' will often be used in combination with the ``annotationTypeProperty''
  attribute.

\item[annotationTypeProperty] if set, the type of each annotation is added to
  the output as this property (i.e. treated as if it were an additional feature
  of the annotation).  This is useful in combination with
  \verb!groupEntitiesBy="set"! when different types of annotation are grouped
  under a single label.

\item[documentAnnotationASName] the annotation set in which to search for a
  \emph{document annotation} (see below).  If omitted, the default set is used.
\item[documentAnnotationType] if specified, the output handler will look for a
  single annotation of this type within the specified annotation set and assume
  that this annotation spans the ``interesting'' portion of the document.  Only
  the text and annotations covered by this annotation will be output, and
  furthermore the features of the document annotation will be added as
  top-level properties (alongside ``text'' and ``entities'') of the generated
  JSON object.  This option is intended to support round-trip processing of
  documents that were originally loaded from JSON by GATE's Twitter support.
\ede

The \verb!JSONStreamingOutputHandler! writes the same JSON format, but instead
of storing each GATE document in its own individual file on disk, this handler
creates one large file (or several ``chunks'') and writes documents to this
file in one stream, separated by newlines.  In addition to the parameters
described above this handler adds two further parameters:

\bde
\item[pattern] (optional, default \verb!part-%03d!) the pattern on which chunk
  file names should be created.  This is a standard Java \verb!String.format!
  pattern string which will be instantiated with a single integer parameter, so
  should include a single \verb!%d!-based placeholder.  Output file names are
  generated by instantiating the pattern with successive numbers starting from
  0 and passing the result to the configured naming strategy until a file name
  is found that does not already exist.  With the default naming strategy this
  effectively means \verb!{dir}/{pattern}{fileExtension}!, e.g.
  \verb!output/part-003.json.gz!
\item[chunkSize] (optional, default \verb!99000000!) approximate maximum size
  in bytes of a single output file, after which the handler will close the
  current file and start the next chunk.  The file size is checked after every MB
  of uncompressed data, so each chunk should be no more than 1MB larger than the
  configured chunk size.  The default chunkSize is 99 million bytes, which
  should produce chunks of no more than 100MB.
\ede

This handler, like the \verb!JSONStreamingInputHandler! can cope with a wider
variety of compression formats than the standard one-file-per-document output
handlers.  A value other than ``none'' or ``gzip'' for the ``compression''
parameter will be taken as the command line for a native compression program
that expects raw data on its stdin and produces compressed data on stdout, for
example \verb!"bzip2"! or \verb!"lzop"! (with the default naming strategy, the
configured fileExtension should take the compression format into account, e.g.
\verb!".json.lzo"!).

\subsection{The M\'{i}mir Output Handler}

GCP also provides \verb!gate.cloud.io.mimir.MimirOutputHandler! to send annotated documents to a M\'{i}mir server for indexing.  This handler supports the following \verb!<output>! attributes:

\bde
\item[indexUrl] (required) the {\em index URL} of the target index.  See the
  M\'{i}mir documentation for details.
\item[uriFeature] (optional) M\'{i}mir requires a URI to identify each
  document.  This attribute tells GCP that the URI for a document should be
  taken from the document feature with this name.
\item[namespace] (optional) if \verb!uriFeature! is not specified GCP will
  construct a suitable URI by appending the document ID to a fixed
  ``namespace'' string.  If omitted an empty namespace will be used (i.e. the
  URI passed to M\'{i}mir will be just the document ID).
\item[username] (optional) HTTP basic authentication username to pass to the
  M\'{i}mir index.  If omitted, no authentication token will be passed.
\item[password] (required if and only if username is specified) the
  corresponding basic authentication password.
\item[connectionInterval] the default behaviour of the M\'{i}mir output handler
  is to open a new connection for each document. When processing very short
  documents, such as tweets, or paper abstracts, using many parallel threads, 
  it is possible that new connections will be opened several hundred times a
  second. This has the potential to overload the receiving M\'{i}mir server, or
  to trigger some security measures, leading to refused connections. To avoid 
  this, the output handler can be configured to accumulate the processed 
  documents in memory, and only connect to the remote server from time to time, 
  sending all the pending documents each time. To enable this functionality, set 
  the value for the {\tt connectionInterval} to a positive value representing 
  the number of milliseconds between connections. For example, a setting of 
  {\tt 1000} means that a new connection will be opened every second.
\ede

This handler ignores annotation set filters -- the complete document will be
sent to M\'{i}mir.

\subsection{Conditional Output}\label{sec:batch-def:conditional-output}
All output handlers support conditional output: the option to only save
some of the documents, based on the value of a document feature. To make use of
this facility, you need to specify which document feature should be read when
deciding whether or not to save a given document, by adding an attribute named
\verb!conditionalSaveFeatureName! to the \verb!output! XML tag in the batch
definition, like in the following example:

\begin{lstlisting}[language=XML]
<output conditionalSaveFeatureName="save"
    dir="../output-files-gate"
    compression="gzip"
    encoding="UTF-8"
    fileExtension=".GATE.xml.gz"
    class="gate.cloud.io.file.GATEStandOffFileOutputHandler" />
\end{lstlisting}

In this example, for each processed document, a document feature named {\tt
save} will be sought. If this is found, and if its value is logical true (i.e.
the feature value is a String with the content `{\it true}', `{\it yes}', or
`{\it on}', regardless of case) then the document will be saved, otherwise it
will be ignored.

\section{Specifying the Documents to Process}

If you are not using a streaming input handler then the final section of the
batch definition specifies which document IDs GCP should process.  The IDs can
be specified in two ways:

\bit
\item Directly in the XML as \verb!<id>doc/id/here</id>! elements.
\item By defining a {\em document enumerator} which generates a list of IDs
  from some external source.
\eit

GCP provides document enumerator implementations corresponding to the default
input handlers, so a typical batch with a ZIP input handler, for example, would
use a ZIP enumerator (configured for the same ZIP file) to generate the list of
document IDs.

A document enumerator is configured using a \verb!<documentEnumerator>! XML
element inside the \verb!<documents>! element.  As with input and output
handlers, this element requires a \verb!class! attribute specifying the Java
class of the enumerator implementation and other attributes are handed off to
the enumerator object to configure itself.

\subsection{The File and ZIP enumerators}

The default enumerator implementation corresponding to the file and ZIP input
handlers are closely related to one another.

The \verb!gate.cloud.io.file.FileDocumentEnumerator! takes a \verb!dir!
attribute and the \verb!gate.cloud.io.zip.ZipDocumentEnumerator! takes
\verb!srcFile! and \verb!fileNameEncoding! attributes (as described above for
their corresponding input handlers) specifying where to find the directory
or ZIP file to be enumerated.  To define which files (or ZIP entries) to
enumerate, the enumerators use the ``fileset'' abstraction from Apache Ant,
controlled by the following attributes:

\bde
\item[includes] (optional) comma-separated file name patterns specifying which
  files to include in the search, e.g. \verb!"**/*.xml,**/*.XML"!.  If
  omitted, all files or ZIP entries are included.
\item[excludes] (optional) comma-separated file name patterns specifying which
  files to exclude from the search, e.g. \verb!"**/*.ignore.xml"!.  If omitted,
  nothing is excluded.
\item[defaultExcludes] (optional) Ant filesets exclude certain file patterns by
  default (\url{http://ant.apache.org/manual/dirtasks.html#defaultexcludes}),
  and the GCP enumerators behave likewise.  To {\em disable} the default
  excludes, set this attribute to ``off'' or ``no''.
\item[prefix] (optional) a prefix to prepend to the paths that are returned by
  the fileset.  For example, if a batch has a file input handler pointing to
  the directory \verb!/data! and an enumerator pointing to the directory
  \verb!/data/large! then the enumerator would need a prefix of ``large/'' to
  produce IDs that are meaningful to the input handler.
\ede

See the Ant documentation for full details on the include and exclude patterns
supported by filesets.  The IDs returned by the enumerator will be those that
match at least one of the include patterns and also do not match any of the
exclude patterns.

Note also that include and exclude patterns are {\em case sensitive}, so a
pattern of ``*.xml'' would not match ``FILE.XML'', for example.  To match both
upper and lower-case variants, include both forms in the pattern.

\subsection{The ARC and WARC enumerators}

The \verb!gate.cloud.io.arc.ARCDocumentEnumerator! and
\verb!WARCDocumentEnumerator! classes enumerate entries in an
ARC or WARC file, and would typically be used in conjunction with the
corresponding input handler.  The enumerators support the following attributes:

\bde
\item[srcFile] (required) the path to the archive to enumerate.
\item[mimeTypes] (optional) whitespace-separated list of MIME types.  If
  specified, the enumerator will only include entries in the archive whose
  header specifies one of the given MIME types.  So a value of
  \verb!"text/html application/pdf"! would enumerate only HTML and PDF files
  from the archive.
\item[includeStatusCodes] (optional) Each entry in an archive file records the
  HTTP status code (200, 301, 404, etc.) that was returned by the server when
  the item was crawled.  This attribute gives a regular expression that is
  matched against the status codes that should be included in the enumeration.
  If omitted, all status codes are included (except those excluded by
  \verb!excludeStatusCodes!).
\item[excludeStatusCodes] (optional) regular expression giving the status codes
  that should be excluded from the enumeration.  If {\em both}
  \verb!includeStatusCodes! {\em and} \verb!excludeStatusCodes! are omitted,
  the default behaviour is to assume an exclude pattern of \verb![345].*! (i.e.
  omit all 3xx, 4xx and 5xx status codes).
\ede

The enumerators returns document IDs in the form required by the corresponding
handlers:

\begin{lstlisting}[language=XML]
<id recordPosition="{zero-based index into the archive}"
    recordOffset="{byte offset of the start of this record}"
    recordLength="{length of the record in bytes}"
    >{original URL from which the document was crawled}</id>
\end{lstlisting}

This format is designed to work well in combination with the
\verb!ARCDocumentNamingStrategy!  for file-based output handlers.

\subsection{The {\tt ListDocumentEnumerator}}

\verb!gate.cloud.io.ListDocumentEnumerator! is the final enumerator
implementation provided by default, and it simply reads a list of document IDs
from a plain text file, one ID per line.  It supports the following attributes:

\bde
\item[file] (required) the location of the text file.
\item[encoding] (optional) the character encoding of the file.  If omitted, the
  platform default encoding is assumed.
\item[prefix] (optional) a common prefix to prepend to each ID (see the
  file-based enumerator for an example of this).
\ede

This enumerator treats each line of the specified file as a separate document
ID, ignoring leading and trailing whitespace on the line.  It is intended for
use when there is a separate process (such as a Perl script) that generates the
list of IDs in advance.


% vim:ft=tex
