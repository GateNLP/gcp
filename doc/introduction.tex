{\em GCP} is a tool designed to support the execution of pipelines built using
GATE Developer over large collections of thousands or millions of documents,
using a multi-threaded architecture to make the best use of today's multi-core
processors.  GCP tasks or {\em batches} are defined using an extensible XML
syntax, describing the location and format of the input files, the GATE
application to be run, and the kinds of outputs required.  A number of standard
input and output handlers are provided, but all the various components are
pluggable so custom implementations can be used if the task requires it.
GCP keeps track of the progress of each batch in a human- and machine-readable
XML format, and is designed so that if a running batch is interrupted for any
reason it can be re-run with the same settings and GCP will automatically
continue from where it left off.

\section{Definitions}\label{sec:intro:definitions}

This section defines a number of terms that have specific meanings in GCP.

\subsection*{Batch}

A {\em batch} is the unit of work for a GCP process.  It is described by an XML
file, and includes the location of a saved GATE application state (a ``gapp''
file), the location of the {\em report} file, one {\em input handler}
definition, zero or more {\em output handler} definitions and a specification
of which documents from the input handler should be processed (either as an
explicit list of document IDs or as a {\em document enumerator} which
calculates the IDs in an appropriate manner).

Chapter~\ref{chap:batch-def} describes the format of batch definition files in
detail.

\subsection*{Report}

The progress of a running batch is recorded in an XML {\em report} file as the
documents are processed.  For each document ID, the report records whether the
document was processed successfully or whether processing failed with an error.
For successful documents the report includes statistics on how many annotations
were found in the document, and for a completed batch it also records overall
statistics on the number of documents and total amount of data processed, the
total number of successful and failed documents and the total processing time.

The report file for a batch is also the mechanism which allows GCP to recover
if processing is unexpectedly interrupted.  If GCP is asked to process a batch
where the report file already exists it will parse the existing report and
ignore documents that are marked as having already been successfully processed.
Thus you can simply restart a crashed GCP batch with the same command-line
settings and it will continue processing from where it left off on the previous
run.

\subsection*{GATE application}

A GCP batch specifies the GATE application that is to be run over the documents
as a standard ``GAPP file'' saved application state, which would typically be
created using GATE Developer.

\subsection*{Input handler}

The {\em input handler} for a GCP batch specifies the source of documents to be
processed.  The job of an input handler is to take a document {\em ID} and load
the corresponding GATE \verb!Document! object ready to be processed.  There are
a number of standard input handlers provided with GCP to take input documents
from individual files on disk, directly from a ZIP archive file or from an ARC
file as produced by the Heritrix web crawler
(\url{http://crawler.archive.org}).  If the standard handlers do not suit your
needs then you can provide a custom implementation by including your handler
class in a GATE CREOLE plugin referenced by your saved application.

\subsection*{Document enumerator}

While the input handler specifies how to go from document IDs to
\verb!gate.Document! objects, it does not specify which document IDs are to be
processed.  The IDs can be specified explicitly in the batch XML file but more
commonly an {\em enumerator} would be used to build a list of IDs by scanning
the input directory or archive file.  Standard enumerator implementations are
provided, corresponding to the standard input handler types, to select a subset
of documents from the input directory or archive according to various criteria.
As with input handlers, custom enumerator implementations can be provided
through the standard CREOLE plugin mechanism.

\subsection*{Output handler}

Most batch definitions will include one or more {\em output handler}
definitions, which describe what to do with the document once it has been
processed by the GATE application.  Standard output handler implementations are
provided to save the documents as GATE XML files, plain text and XCES standoff
annotations, inline XML (``save preserving format'' in GATE Developer terms),
and to send the annotated documents to a M\'{i}mir server for indexing.
Custom implementations can be added using the CREOLE plugin mechanism.

Note that output handler definitions are optional -- if you do not specify any
output handlers then GCP will not save the results anywhere, but this may be
appropriate if, for example, your pipeline contains a custom PR that saves your
results to a relational database or similar.


\section{Processing Model}

GCP processes a batch as follows.

\ben
\item Parse the batch definition file, and run the document enumerators (if
  any) to build the complete list of document IDs to be processed.
\item Parse the existing (possibly partial) report file, if one exists, and
  remove from this list any documents that are already marked as having been
  successfully processed.
\item Create a thread pool of a size specified on the command line (the default
  is 6 threads).
\item Load the saved application state, and use
  \htlink{http://gate.ac.uk/gate/doc/javadoc/gate/Factory.html\#duplicate(gate.Resource)}%
  {\tt Factory.duplicate} to make additional copies of the application such
  that there is one independent copy of the application per thread in the
  thread pool.
\item Run the processing threads.  Each thread will repeatedly: 
  \bit
  \item take the next available unprocessed document ID from the list.
  \item ask the input handler for the corresponding \verb!gate.Document!.
  \item put that document into a singleton \verb!Corpus! and run this thread's
    copy of the GATE application over that corpus.
  \item pass the annotated document to each of the output handlers.
  \item write an entry to the report file indicating whether the document was
    processed successfully or whether an exception occurred during processing.
  \item release the document using \verb!Factory.deleteResource!.
  \eit
\item Once all the documents have been processed, shut down the thread pool and
  call \verb!Factory.deleteResource! to cleanly shut down the GATE
  applications.
\een

Due to the asynchronous nature of the processing threads, if one document takes
a particularly long time to process the other threads can proceed with many
other documents in parallel, they are not forced to wait for the slowest
thread.

\section{Changelog}\label{sec:intro:changes}

This section summarises the main changes between releases of GCP

\subsection{3.2 (May 2021)}

Underlying GATE Embedded version is now 9.0.1.  This means that all logging
within GCP has been switched from Log4J to SLF4J in line with the same change
in GATE Embedded; if you have edited the default logging configuration you
will need to port your changes to Logback format.

\subsection{3.1.1 (January 2020)}

The only change in this release is to upgrade the GATE Embedded version to
8.6.1.

\subsection{3.1 (June 2019)}

Underlying GATE Embedded version is now 8.6.  Also upgraded commons-compress
dependency (due to a security advisory).

Note \verb!gcp-direct! mode no longer loads the FastInfoset format plugin
automatically -- if you require FastInfoset you must use a saved application
that includes the format plugin, or provide your own \verb!-p! option to load
your preferred version, e.g.
\begin{verbatim}
-p uk.ac.gate.plugins:format-fastinfoset:8.5
\end{verbatim}

\subsection{3.0.1 (September 2018)}

Minor bug fixes, and a couple of dependency upgrades to be compatible with the
latest version of the Twitter document format plugin.  Underlying GATE Embedded
version is now 8.5.1.

\subsection{3.0 (May 2018)}

Updated to work with GATE Embedded version 8.5:
\bit
\item Minimum required Java version is now Java 8.
\item GCP now builds using Maven rather than Ant, and has been split into
  ``api'' and ``impl'' modules. GATE plugins that want to provide custom input
  or output handlers should declare a ``provided'' dependency on the
  appropriate version of \verb!uk.ac.gate:gcp-api!.
\item New command line parameters \verb!-C! and \verb!-p! to allow pre-loading
  of specific GATE plugins in addition to those declared by the saved
  application.  This is useful, for example, to load plugins that provide
  document format parsers.
\eit

\subsection{2.8.1 (June 2017)}

GCP now depends on GATE Embedded 8.4.1.  Also the \verb!-i! option to
\verb!gcp-direct.sh! can now be a \emph{file} which lists the documents to
process, instead of just a \emph{directory} of documents.

\subsection{2.8 (February 2017)}

GCP now depends on GATE Embedded 8.4.

\subsection{2.7 (January 2017)}

This is a minor bugfix release, the main change is that GCP now depends on GATE
Embedded 8.3.  There have been minor changes to the \verb!gcp-direct.sh! script,
in particular it is now possible to run a pipeline with \emph{no} output
handler at all, useful in cases where there is a PR within the pipeline that is
responsible for handling the output, or if you want to run a pipeline purely
for its side effects (e.g. building a frequency table or training some sort of
machine learning model).

\subsection{2.6 (June 2016)}

This is a minor bugfix release, the main change is that GCP now depends on GATE
Embedded 8.2.

\subsection{2.5 (June 2015)}

\bit
\item Now depends on GATE Embedded 8.1
\item Introduced ``streaming'' style input and output handlers for JSON
  data (e.g. from Twitter), which can read a series of documents from
  a single JSON input file, and write JSON results to a single concatenated
  output file (sections~\ref{sec:batch-def:json-input} and
  \ref{sec:batch-def:file-output-handlers}).
\item Introduced the \verb!gcp-direct.sh! script to cover simple invocations
  of GCP without the need to write a batch definition XML file
  (section~\ref{sec:running:gcp-direct}).
\item For ``controller-aware''
  PRs\footnote{\url{http://gate.ac.uk/gate/doc/javadoc/gate/creole/ControllerAwarePR.html}},
  the various callbacks are now invoked just once per batch rather than before
  and after every single document.
\eit

\subsection{2.4 (May 2014)}

\bit
\item Now depends on GATE Embedded 8.0 (and thus requires Java 7 to run)
\item Added input handler for WARC format archives, to complement the existing
  ARC handler (section~\ref{sec:batch-def:arc}).
\item ARC and WARC handlers can optionally load individual records from
  remotely hosted archives using HTTP requests with a ``Range'' header.  This
  facility can be used with publicly-hosted data sets such as Common
  Crawl\footnote{\url{http://www.commoncrawl.org}}.  To support this
  functionality, document identifiers in a batch definition can now take XML
  attributes as well as the actual string identifier (exactly how such
  attributes are used is up to the handler implementations).
\item Added output handler to save documents in a JSON format modelled on that
  used by Twitter to represent ``entities'' (e.g. username mentions) in Tweets.
\item Efficiency improvements in the M\'{i}mir output handler, to send
  documents to the server in batches rather than opening a new HTTP connection
  for every document.
\eit

\subsection{2.3 (November 2012)}

\bit
\item Now depends on GATE Embedded 7.1
\item Introduced support for \emph{conditional saving of documents}
  (section~\ref{sec:batch-def:conditional-output})
\item Added the \emph{serialized object output handler}
  (section~\ref{sec:batch-def:file-output-handlers})
\item More robust and reliable counting of the size of each input document.
\eit

\subsection{2.2 (February 2012)}

\bit
\item Now depends on GATE Embedded 7.0
\item Introduced Java-based command line interface to replace the \verb|gcp.sh|
  shell script, which behaves more consistently across platforms.
\eit

% vim:ft=tex
