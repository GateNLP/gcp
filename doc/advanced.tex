\section{GATE Configuration}

GCP uses its own GATE site and user configuration files, located in the
\verb!gate-home! directory under the GCP installation root directory.  It
deliberately does not use the user's own \verb!~/.gate.xml!, so if you need to
configure any options that must be set through configuration files you will
need to edit \verb!gate-home/user-gate.xml!.  The most likely configuration
option that you will need to set this way is whether or not to add additional
spaces to separate otherwise adjacent text in different XML or HTML elements
when unpacking markup.  This is the \verb!Document_add_space_on_unpack! option
in \verb!user-gate.xml!.

\section{JMX Monitoring}

The GCP batch runner registers an MBean with the platform JMX MBean server in
its JVM that makes it possible to query the state of the running batch from the
JMX management console or (using the standard JMX APIs) from another Java
process.  The process of connecting a JMX client to the GCP process is beyond
the scope of thie guide, here we simply describe the MBean interface and the
attributes it exposes to clients.

The MBean implements the \verb!gate.cloud.batch.BatchJobData! interface which
is defined in the \verb!gcp-api! JAR file\footnote{If your application requires
this dependency solely to be able to communicate with a running GCP over JMX then
you can safely exclude the transitive dependency on \texttt{gate-core}.}:

\begin{lstlisting}[breaklines]
public JobState getState();
public int getProcessedDocumentCount();
public int getTotalDocumentCount();
public int getRemainingDocumentCount();
public int getSuccessDocumentCount();
public int getErrorDocumentCount();
public long getStartTime();
public String getBatchId();
public long getTotalDocumentLength();
public long getTotalFileSize();
\end{lstlisting}

The \verb!getState()! method returns the current state of the batch.  The state
is an enum type, and will usually be \verb!JobState.RUNNING!.  The various
\verb!get*DocumentCount! methods provide access to the number of documents that
have been processed, the number of those that were successful/failed and the
number remaining to be processed.  This allows the monitoring process to check
that the GCP process is not ``stuck''.  The \verb!getTotalDocumentLength! and
\verb!getTotalFileSize! methods give the total length of the documents
processed (in plain text characters, after GATE has unpacked the markup) and
the total size (in bytes) of the files processed so far.

% vim:ft=tex
