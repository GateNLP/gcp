\section{GATE Configuration}

By default GCP uses its own GATE user configuration file, located in the
\verb!gate-home! directory under the GCP installation root directory.  It
deliberately does not use the user's own \verb!~/.gate.xml!, so if you need to
configure any options that must be set through configuration files you will
need to edit \verb!gate-home/user-gate.xml!.  The most likely configuration
option that you will need to set this way is whether or not to add additional
spaces to separate otherwise adjacent text in different XML or HTML elements
when unpacking markup.  This is the \verb!Document_add_space_on_unpack! option
in \verb!user-gate.xml!.

If you want to use your own user configuration you can do so by setting the
\verb|gate.user.config| system property.

\section{Maven settings}\label{sec:maven-settings}

GATE (and thus GCP) use the standard Maven configuration mechanism to define
things like the location of the ``local repository'' cache folder, additional
Maven repositories to search for plugins, ``mirror'' repositories, HTTP proxy
configuration, etc.  In the absence of any specific configuration GATE will
look for plugins in the standard Central Repository, as well as the GATE team
repository at \verb!http://repo.gate.ac.uk! (which is where nightly snapshot
builds of the standard plugins are deployed), and will use the default Maven
local repository cache in the folder \verb!.m2/repository! in the home
directory of the user running GCP.  Additional configuration can be provided
via a Maven settings file, which is read by default from
\verb!.m2/settings.xml! but this location can be overridden by passing a system
property \verb!M2_SETTINGS_XML!:
\begin{lstlisting}[breaklines]
java -jar gcp-cli.jar -DM2_SETTINGS_XML=m2/settings.xml -t 4 -m 2G -d control
\end{lstlisting}

The \verb!settings.xml! file is a standard Maven format documented at
\url{https://maven.apache.org/settings.html}, the settings most useful to GCP
are:
\begin{itemize}
\item \verb!localRepository! -- specify an alternative local repository cache
  location instead of using one in the user's home directory.
\item \verb!offline! -- if set to \verb!true!, this tells GATE not to attempt
  to resolve any dependencies from the internet.  If the app refers to a plugin
  that is not cached locally (in either the \verb!localRepository!, a cache
  folder specified with the \verb!-C! option on the GCP command line, or a
  \verb!maven-cache.gate!  folder alongside the saved application) then GCP
  will fail to load the app.  This option can be useful when running
  applications that were saved using the ``export for GATE Cloud'' tool in GATE
  Developer, as this is intended to make them completely self-contained.
\item \verb!proxies! -- can be used to configure HTTP proxy settings if these
  are required to access remote repositories on the internet, however GATE also
  respects the standard Java system properties for proxy configuration so it is
  usually better to configure proxying using these rather than via Maven
  settings.
\end{itemize}

\section{JMX Monitoring}

The GCP batch runner registers an MBean with the platform JMX MBean server in
its JVM that makes it possible to query the state of the running batch from the
JMX management console or (using the standard JMX APIs) from another Java
process.  The process of connecting a JMX client to the GCP process is beyond
the scope of thie guide, here we simply describe the MBean interface and the
attributes it exposes to clients.

The MBean implements the \verb!gate.cloud.batch.BatchJobData! interface which
is defined in the \verb!gcp-api! JAR file\footnote{If your application requires
this dependency solely to be able to communicate with a running GCP over JMX then
you can safely exclude the transitive dependency on \texttt{gate-core}.}:

\begin{lstlisting}[breaklines]
public JobState getState();
public int getProcessedDocumentCount();
public int getTotalDocumentCount();
public int getRemainingDocumentCount();
public int getSuccessDocumentCount();
public int getErrorDocumentCount();
public long getStartTime();
public String getBatchId();
public long getTotalDocumentLength();
public long getTotalFileSize();
\end{lstlisting}

The \verb!getState()! method returns the current state of the batch.  The state
is an enum type, and will usually be \verb!JobState.RUNNING!.  The various
\verb!get*DocumentCount! methods provide access to the number of documents that
have been processed, the number of those that were successful/failed and the
number remaining to be processed.  This allows the monitoring process to check
that the GCP process is not ``stuck''.  The \verb!getTotalDocumentLength! and
\verb!getTotalFileSize! methods give the total length of the documents
processed (in plain text characters, after GATE has unpacked the markup) and
the total size (in bytes) of the files processed so far.

% vim:ft=tex
