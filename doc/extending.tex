GCP has been designed to be easily extensible.  Additional input/output
handlers, naming strategies and document enumerators can be added by writing a
Java class that implements the relevant interface and placing that class in a
CREOLE plugin loaded by your application.  The class can then be named in your
batch definitions as appropriate, and GCP will load it from the GATE
ClassLoader.

\section{Custom Input Handlers}\label{sec:extending:input}

Input handlers must implement the \verb!gate.cloud.io.InputHandler! interface:

\begin{lstlisting}[breaklines]
public void config(Map<String, String> configData) throws IOException, GateException;
public void init() throws IOException, GateException;
public void close() throws IOException, GateException;

public Document getInputDocument(String id) throws IOException, GateException;
\end{lstlisting}

The handler class must also have a no-argument constructor.  When GCP parses
the batch definition it creates an instance of the handler class using that
constructor, then calls the \verb!config! method, passing in a map containing
the XML attribute values from the \verb!<input>! element in the batch file.  An
additional ``virtual'' attribute named
\verb!batchFileLocation!\footnote{Available as a constant
{\tt gate.cloud.io.IOConstants.PARAM\_BATCH\_FILE\_LOCATION}} is also available
in this map, containing the path to the XML batch definition file itself.  This
is made available to allow the handler to resolve relative path expressions in
other attribute values against the location of the XML file (see the standard
handler implementations for examples of this).  The \verb!config! method should
read any required parameters from the config map, and should throw a suitably
descriptive \verb!GateException! if any required parameters are missing.

Next, GCP will call the handler's \verb!init()! method.  This method is
responsible for setting up the handler, and is where you should open any
required external resources such as index files.  Both \verb!config! and
\verb!init! are guaranteed to be called sequentially in a single thread.

Conversely, \verb!getInputDocument! will be called by the processing threads,
and must therefore be thread-safe.  However you should avoid locking or
\verb!synchronized! blocks within \verb!getInputDocument! if at all possible,
so as not to block processing threads against one another.  This method is the
one responsible for actually loading the GATE Document corresponding to a given
document ID.  The handler does not need to (indeed should not) retain a
reference to the document that it returns, as the processing thread is
responsible for freeing the document with \verb!Factory.deleteResource! when
processing is complete.

Finally, the \verb!close! method will be called (in a single thread) when the
entire batch is complete.  This is where you should release resources that were
acquired in the \verb!init! method.

It is good manners, though not strictly required, for input handlers to provide
a meaningful \verb!toString()! implementation.

\section{Custom Output Handlers}

Output handlers must implement the \verb!gate.cloud.io.OutputHandler!
interface:

\begin{lstlisting}[breaklines]
public void setAnnSetDefinitions(List<AnnotationSetDefinition> annotationsToSave);
public List<AnnotationSetDefinition> getAnnSetDefinitions();

public void config(Map<String, String> configData) throws IOException, GateException;
public void init() throws IOException, GateException;
public void close() throws IOException, GateException;

public void outputDocument(Document document, String documentId) throws IOException, GateException;
\end{lstlisting}

GCP will instantiate the handler class and call \verb!config! exactly as for
input handlers (see section~\ref{sec:extending:input}).  It will then call
\verb!setAnnSetDefinitions! to pass in the annotation set definitions specified
by the \verb!<annotationSet>! elements in the XML (if any), and then call
\verb!init!.  As before these three methods are called in sequence in a single
thread.

As documents are processed, the various processing threads will call
\verb!outputDocument!, passing an annotated document as a parameter.  This
method must therefore be thread-safe, but should avoid synchronization and
locking if at all possible.

Finally, the \verb!close! method will be called (in a single thread) when the
entire batch is complete.  This is where you should release resources that were
acquired in the \verb!init! method.

It is good manners, though not strictly required, for output handlers to
provide a meaningful \verb!toString()! implementation.

GCP provides an abstract base class \verb!gate.cloud.io.AbstractOutputHandler!
which custom output handler implementations may choose to extend.  This class
provides an implementation of the \verb!get! and \verb!setAnnSetDefinitions!
methods, and a utility method to extract the annotations corresponding to these
definitions from an annotated document at output time.

\section{Custom Naming Strategies}

Custom naming strategies (for use with the file-based input and output
handlers) must implement the \verb!gate.cloud.io.file.NamingStrategy!
interface:

\begin{lstlisting}[breaklines]
public void config(boolean isOutput, Map<String, String> configData) throws IOException, GateException;

public File toFile(String id) throws IOException;
\end{lstlisting}

The \verb!config! method is similar to the equivalent method on input and
output handlers, but it has an extra parameter indicating whether this naming
strategy is being used by an input (false) or output (true) handler.  This
allows the strategy to adjust its behaviour as appropriate, for example in the
input case it is an error if the specified base directory does not exist,
whereas for output this is permitted as the output handler will create
intermediate directories as required.

The \verb!toFile! method will be called from processing threads by the input
(or output) handler that uses this strategy, and should return the
\verb!java.io.File! that corresponds to the given document ID.


\section{Custom Document Enumerators}

Document enumerators must implement the \verb!gate.cloud.io.DocumentEnumerator!
interface:

\begin{lstlisting}[breaklines, texcl]
public void config(Map<String, String> configData) throws IOException, GateException;
public void init() throws IOException, GateException;

// DocumentEnumerator extends java.util.Iterator$<$String$>$
public boolean hasNext();
public String next();
public void remove();
\end{lstlisting}

They have the familiar \verb!config! and \verb!init! methods which are called
as for input and output handlers (see section~\ref{sec:extending:input}).  To
actually enumerate documents GCP uses the standard \verb!Iterator<String>!
methods, the enumerator is expected to return one ID from each call to
\verb!next!.  The \verb!remove! method is specified by the Iterator interface
but will never be called by GCP, the standard enumerators all implement this
method to throw an \verb!UnsupportedOperationException!.

% vim:ft=tex
